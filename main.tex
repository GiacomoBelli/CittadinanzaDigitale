\documentclass{article}
\usepackage[utf8]{inputenc}
\usepackage{makeidx}
\makeindex
\title{Cittadinanza digitale e tecnocivismo "Fake news"}
\author{Belli Giacomo }
\date{Anno 2021/2022}
\begin{document}
\maketitle
\section{Introduzione}
In questo articolo esporrò il processo, storico e moderno, con il quale articoli e notizie vengono creati e come questi influenzano sia il processo di informatizzazione dei cittadini sia la potenza mediatica con il quale queste notizie sono in grado di cambiare il pensiero delle persone riguardo un determinato argomento.
Il primo avvento documentato riguardo alla stampa è con l'uscita del libro "La bibbia di Gutenberg" nel 1455, grazie alla tecnica dei caratteri mobili.
Inizialmente la stampa venne usata per diffondere per conto della chiesa il libro più diffuso al mondo ovvero la "Bibbia", soltanto in seguito le varie istituzioni governative hanno utilizzato la stampa come mezzo di propaganda per informare i cittadini riguardo alle vicende del paese e per accrescere il consenso popolare verso il proprio governo.
Solo nel 1847 con l'avvento dello "Statuto Albertino" ci si avvicinò alla libertà di stampa e informazione di cui godiamo oggi.
Tuttavia all'epoca le notizie venivano spesso ritoccate o inventate completamente, con lo scopo di influenzare la popolazione riguardo determinati argomenti.
Oggi la situazione è decisamente cambiata anche se le notizie false sono ancora in circolazione, tuttavia a differenza del passato in cui il governo stesso era l'artefice delle falsità che venivano messe in circolazione, oggi abbiamo enti e istituzioni che possono verificare la veridicità dei documenti e posso attestare la natura e la sicurezza di quest'ultimi ad esempio "OMS", l'associazione mondiale della sanità o come "ITU", l'organizzazione internazionale per le telecomunicazioni.
\section{Lv 0}
\subsection{Informazioni in formato digitale}
Come è cambiato il modo di fare notizia con l'arrivo del digitale, informazioni digitalizzate, mass media privati e del governo e il ruolo che ha avuto internet.
\subsection{"Fake news" over internet}
Ruolo della rete nello storing and sharing di informazioni.
\subsection{Mezzi di propagazione}
Principali mezzi di propagazione di notizie, informazioni tramite la rete, informazioni real time e tecnologie.
\section{Lv 1}
\subsection{Accessibilità dell'informazione}
Metodi di accesso alle notizie da parte del cittadino, requisiti e limitazioni.
\subsection{Sicurezza e verifica delle informazioni}
Capire chi si occupa di fare informazione e come avviene la verifica delle notizie.
Mezzi che ha a disposizione il cittadino per verificare le infromazioni nella rete.
\section{Lv 2}
\subsection{Diritto di informazione del cittadino}
Diritto all'informazione, legislazione italiana e importanza di essere aggiornati sulle vicende del paese e nel mondo.
\subsection{Servizi pubblici di informazione}
Servizi a cui può accedere il cittadino per l'informazione e QoS (Quality of Services)
\subsection{Identità digitale e SPID}
Capire come funziona e il potenziale dell'identità digitale per la veridicità dei dati, campionamenti o studi che si effettuano.
\subsection{Lv 3}
\subsection{Educazione del cittadino nell'utilizzo di internet}
Educazione del cittadino all'utilizzo delle nuove tecnologie e della rete come mezzo per la propria informazione
\subsection{Risposta delle istituzioni}
Che ruolo hanno le istituzioni per aiutare il cittadino riguardo la veridicità delle informazioni mediante la trasparenza.
\subsection{Come smascherare le "fake news"}
Tecniche immediate e specifiche (Skills) che si possono insegnare al cittadino per validare la veridicità di notizie e documenti amministrativi e non.
\subsection{Social media come mezzo di informazione e propagazione}
Studio su come i social media hanno superato giornali e televisione come principale mezzo di informazione del cittadino "medio".
\section{Lv 4}
\subsection{Importanza dell'open data}
Importanza dell'open data come incentivo alla trasparenza verso i cittadini o semplicemente agli utenti che usufruiscono di un determinato servizio, applicazione, tecnologia per la propria sicurezza e privacy. 
\subsection{Codice amministrativo digitale CAD}
CAD come mezzo legislativo per documenti e informazioni in formato digitale.
\subsection{Firma digitale nei documenti}
Discutere dell'importanza della firma digitale dei documenti, specialmente per quelli di natura scientifica.
Si possono fare diversi esempi anche sulla speculazione di "fake news" riguardo studi che non sono mai stai approvati dalla comunità scientifica, che non erano firmati (almeno non con l'approvazione di un particolare dipartimento di ricerca) durante la pandemia COVID-19
\section{Conclusioni}
\printindex
\end{document}